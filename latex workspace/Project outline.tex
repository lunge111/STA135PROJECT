\documentclass[12pt,a4paper]{article}
\usepackage{multirow}
\usepackage{bm}
\usepackage{AMSFONTS}
\usepackage{amssymb}
\usepackage{latexsym}
\usepackage{graphicx}
\usepackage{subfigure}
\usepackage[colorlinks,linkcolor=red]{hyperref}
\textwidth 6.5in
\textheight 9in
\topmargin 0pt
\linespread{1.5}
\oddsidemargin 0pt
\begin{document}
\title{\huge STA 135 Project Outline}
\newtheorem{coro}{\hskip 2em Corollary}[section]
\newtheorem{remark}[coro]{\hskip 2em Remark}
\newtheorem{propo}[coro]{\hskip 2em  Proposition}
\newtheorem{lemma}[coro]{\hskip 2em Lemma}
\newtheorem{theor}[coro]{\hskip 2em Theorem}
\newenvironment{prf}{\noindent { proof:} }{\hfill $\Box$}
\date{1/24/2014}
\maketitle
\section{MEMBERS OF OUR GROUP}
\begin{tabular}{lcl}
Yilun Zhang&\qquad&1st year Master of Biostatistics\tabularnewline
Dan Zhang&\qquad&1st year Master of Statistics\tabularnewline
Yu Liu&\qquad&1st year Master of Statistics\tabularnewline
\end{tabular}

\section{BRIEF DESCRIPTION}
\qquad The data comes from an experiment aiming to human activity recognition using Samsung smartphones. The experiments have been carried out with a group of 30 volunteers within an age bracket of 19-48 years. Each person performed six activities (WALKING, WALKING\_UPSTAIRS, WALKING\_DOWNSTAIRS, SITTING, STANDING, LAYING) wearing a smartphone (Samsung Galaxy S II) on the waist. Using its embedded accelerometer and gyroscope, we captured 3-axial linear acceleration and 3-axial angular velocity at a constant rate of 50Hz.

The raw data can be accessed from here:  \\ \url{http://archive.ics.uci.edu/ml/datasets/Human+Activity+Recognition+Using+Smartphones}.



\end{document} 